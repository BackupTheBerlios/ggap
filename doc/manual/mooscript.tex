\chapter{MooScript}

%%%%%%%%%%%%%%%%%%%%%%%%%%%%%%%%%%%%%%%%%%%%%%%%%%%%%%%%%%%%%%%%%%%%%%%%%%%%%%%%%%%
%% Syntax
%%%%%%%%%%%%%%%%%%%%%%%%%%%%%%%%%%%%%%%%%%%%%%%%%%%%%%%%%%%%%%%%%%%%%%%%%%%%%%%%%%%
\section{Syntax}

MooScript is a very simple scripting language made after \python\ and \gap. A
script consists of a sequence of statements separated by semicolons.

\subsection{Values}

The following data types are available in scripts:
\begin{itemize}
\item \none\ - this is a value representing ``no value''.

\item Boolean - boolean type. Values of this type can take two values - \true\ 
and \false.

\item String - string type. Strings in scripts are entered as a sequence of
characters enclosed into single or double quote signs, e.g.
\begin{verbatim}
"a string"
'another string'
\end{verbatim}
String may contain quote sign which is not used for the quoting the string itself,
so one may write \code{"'quoted string'"} or \code{'"another quoted string"'}. This
syntax is useful for producing strings sent to GAP. For example, to send 
a string \code{Print("something");} one just writes \code{GAP('Print("something");')}.

\item Integer - this is the only numerical type in MooScript.

\item List - it is similar to \gap\ lists, with the difference that lists may not have 
gaps. Syntax is the same as in \gap:
\begin{verbatim}
[1,2,3,4]
[1..10]
\end{verbatim}

\end{itemize}

\subsection{Variables}
When script is started, there are few predefined variables, such as \true\ or \none.
New variables are created using assignment operator \code{=}, e.g. \code{a = 1;} or
\code{b = [1,2,3];}. Variables may be assigned new value without restriction, e.g.
\begin{verbatim}
a = 1;
b = 2;
a = a * b + a;
\end{verbatim}

\subsection{Functions}
Function calls have usual syntax \code{function(arg1, arg2, ...);}. All functions
return value (that may be \none), so assignment like \code{a = func(x);} always 
makes sense.

\subsection{Basic operations}

\subsubsection{Operations with integers}

\begin{itemize}
\item \code{-a}
\item \code{a + b}
\item \code{a - b}
\item \code{a * b}
\end{itemize}

\subsubsection{String operations}

\begin{itemize}
\item \code{a + b} - concatenation of strings \variable{a} and \variable{b}.
\item \code{a * n} - for integer \variable{n}, produces string \variable{a} replicated
\variable{n} times.
\item \code{a \% args} - format operator, similar to C printf function.
\begin{verbatim}
'%s%s' % ['a', 'b']
\end{verbatim}
produces 'ab'.

\end{itemize}

\subsubsection{List operations}

\begin{itemize}
\item \code{a + b} - concatenation of lists \variable{a} and \variable{b}.
\end{itemize}

\subsubsection{Logical operations}
Every value produces boolean value \true\ or \false\ as follows:
\begin{itemize}
\item \none\ is \false.
\item a string is \false\ if and only if it's an empty string \code{''}.
\item a list is \false\ if and only if it's an empty list \code{[]}.
\item an integer is \false\ if and only if it's \code{0}.
\end{itemize}

Logical operations accept any arguments, which are converted to booleans
before evaluating logical expression. There are the following logical 
operations:

\begin{itemize}
\item \code{and}. Synonim is \code{\&\&}, e.g. \code{a and b} or \code{a \&\& b}.
\item \code{or}. Synonim is \code{||}, e.g. \code{a or b} or \code{a || b}.
\item \code{not}. Synonim is \code{!}, e.g. \code{not a} or \code{!a}.
\end{itemize}

\subsubsection{Comparison}
Values may be compared. Two values are equal only if they have the same type.
Ordering is defined on sets of values of the same type.
\begin{itemize}
\item \code{a == b} - \true\ if \variable{a} equals \variable{b}. It has usual 
meaning for strings and integers; two lists are equal if and only if they are
equal element-wise.
\item \code{a != b} - negation of \code{==} relation.
\item \code{a < b}
\item \code{a <= b}
\item \code{a > b}
\item \code{a >= b}
\end{itemize}


\subsection{Loops}

There are three kinds of loops.

\code{for \textit{variable} in \textit{list} do \textit{stmt1; stmt2; ... ;} od;}

\code{while \textit{condition} do \textit{stmt1; stmt2; ... ;} od;}

\code{do \textit{stmt1; stmt2; ... ;} while \textit{condition};}


\subsection{Conditionals}

There are two forms of \code{if} statement.

\code{if \textit{condition} then \textit{stmt1; stmt2; ... ;} fi;}

\code{if \textit{condition} then \textit{stmt1; ... ;} else \textit{stmt1; ... ;} fi;}

There is also ternary operator similar to the one in C programming language.

\code{\textit{expr} ? \textit{expr1} : \textit{expr2};}

It evaluates and returns \variable{\textit{expr1}} if \variable{\textit{expr}} 
evaluates to \true, and \variable{\textit{expr2}} otherwise.


%%%%%%%%%%%%%%%%%%%%%%%%%%%%%%%%%%%%%%%%%%%%%%%%%%%%%%%%%%%%%%%%%%%%%%%%%%%%%%%%%%%
%% Builtin values
%%%%%%%%%%%%%%%%%%%%%%%%%%%%%%%%%%%%%%%%%%%%%%%%%%%%%%%%%%%%%%%%%%%%%%%%%%%%%%%%%%%
\section{Builtin values}

These are builtin values always available in scripts.

\variable{none} - the value representing "nothing", or "no value". 
Evaluates as \variable{false} in logical expressions.

\variable{true} - logical true value.

\variable{false} - logical false value.


%%%%%%%%%%%%%%%%%%%%%%%%%%%%%%%%%%%%%%%%%%%%%%%%%%%%%%%%%%%%%%%%%%%%%%%%%%%%%%%%%%%
%% Builtin functions
%%%%%%%%%%%%%%%%%%%%%%%%%%%%%%%%%%%%%%%%%%%%%%%%%%%%%%%%%%%%%%%%%%%%%%%%%%%%%%%%%%%
\section{Builtin functions}

These are builtin functions always available in scripts.

\function{Print}{arg1, ...} - prints its arguments to console. It is
useful for debugging purposes.

\function{Include}{file} - executes content of \arg{file}.

\function{Python}{string} - executes \python\ code from \arg{string}.

\function{Abort}{} - aborts script execution, ``return for poor''.

\function{Str}{val} - converts \arg{val} to a string.

\function{Int}{val} - converts \arg{val} to an integer.

\function{Len}{val} - returns length of \arg{val}. It is character length
for a string or length for a list.

\function{Text}{text = none, dialog\_text = none} - dialog used for displaying 
and modifying big amount of \arg{text}. \returns{new text}.

\function{Entry}{entry\_text = \none, dialog\_text = \none, hide\_text = \false} - 
analogous to \funcname{Text}, but for small amounts of text. 
\returns{new entry content}.

\function{Info}{text} - displays information dialog.

\function{Error}{text} - displays error dialog.

\function{Question}{text} - displays question dialog. \returns{\true\ or \false}.

\function{Warning}{text} - same as \funcname{Question} but with warning icon.

\function{ChooseFile}{title = \none, start = \none} - file chooser dialog.
\returns{filename or \none}.

\function{ChooseFiles}{title = \none, start = \none} - multiple files chooser dialog.
\returns{list of files or \none}.

\function{ChooseDir}{title = \none, start = \none} - directory chooser dialog.
\returns{directory or \none}.

\function{ChooseFileSave}{title = \none, start = \none} - dialog that allows 
choosing existing file or entering new file name.
\returns{filename or \none}.


%%%%%%%%%%%%%%%%%%%%%%%%%%%%%%%%%%%%%%%%%%%%%%%%%%%%%%%%%%%%%%%%%%%%%%%%%%%%%%%%%%%
%% Using Python code
%%%%%%%%%%%%%%%%%%%%%%%%%%%%%%%%%%%%%%%%%%%%%%%%%%%%%%%%%%%%%%%%%%%%%%%%%%%%%%%%%%%
\section{Using \python\ code}

It is possible to use \python\ code in scripts. The syntax used for this is the
following:
\begin{verbatim}
===
python code...
===
\end{verbatim}
Example:
\begin{verbatim}
Info('This is MooScript');
===
import gtk
w = gtk.Window()
l = gtk.Label('and this is Python')
w.add(l)
w.show_all()
===
Info('MooScript again');
\end{verbatim}



%%%%%%%%%%%%%%%%%%%%%%%%%%%%%%%%%%%%%%%%%%%%%%%%%%%%%%%%%%%%%%%%%%%%%%%%%%%%%%%%%%%
%% Examples
%%%%%%%%%%%%%%%%%%%%%%%%%%%%%%%%%%%%%%%%%%%%%%%%%%%%%%%%%%%%%%%%%%%%%%%%%%%%%%%%%%%
\section{Examples}

\subsection{Stupid game}
\begin{verbatim}
stop = false;
attempts = 0;
while not stop do
  word = Entry('', 'Guess a word');
  attempts = attempts + 1;
  if word == 'word' then
    stop = true;
    Info('Yes!');
  else
    Error('Nope');
  fi;
od;
\end{verbatim}

\subsection{Choosing file}
This example uses \code{GAP} command which pastes a string to the \gap\
console.
\begin{verbatim}
file = ChooseFile('Pick a File');
if file then
  GAP('Read("%s");\n' % [file]);
fi;
\end{verbatim}
